\documentclass[lettersize,journal]{IEEEtran}
\usepackage{amsmath,amsfonts}
\usepackage{algorithmic}
\usepackage{algorithm}
\usepackage{array}
\usepackage[caption=false,font=normalsize,labelfont=sf,textfont=sf]{subfig}
\usepackage{textcomp}
\usepackage{stfloats}
\usepackage{url}
\usepackage{graphicx}
\usepackage{cite}

% 可根据需要添加其他宏包
\usepackage{booktabs}
\usepackage{makecell}
\usepackage{siunitx}
\usepackage{hyperref}
\usepackage{tabularx}

\begin{document}

\title{A Resource-Efficient 1024-Point MSC FFT Using Time-Multiplexed Constant Multiplication}

\author{
    First A. Author,~\IEEEmembership{Member,~IEEE,}
    Second B. Author,~\IEEEmembership{Fellow,~IEEE,}
    and Third C. Author,~\IEEEmembership{Student Member,~IEEE}
}

\markboth{IEEE Transactions ON CIRCUITS AND SYSTEMS---II: EXPRESS BRIEFS,~Vol.~XX, No.~XX, Month~Year}%
{Author \MakeLowercase{\textit{et al.}}: A Resource-Efficient 1024-Point MSC FFT Using Time-Multiplexed Constant Multipliers}


\maketitle

\begin{abstract}
This paper presents a resource-efficient architecture for a 1024-point FFT based on time-multiplexed constant multipliers. The proposed design reduces hardware cost while maintaining high throughput and numerical accuracy.
\end{abstract}

\begin{IEEEkeywords}
FFT, constant multiplication, time-multiplexing, FPGA, ASIC, low-complexity, signal processing.
\end{IEEEkeywords}

\section{Introduction}
% Introduction content here

\section{Background and Related Work}
% Background and literature review here

\section{Proposed Architecture}
% Detailed description of your MSC FFT architecture

\section{Implementation and Optimization}
% Hardware implementation details, time-multiplexing strategy, etc.

\section{Experimental Results}
% 该部分展示实验结果,包括资源利用率、SQNR、吞吐量等指标的比较
Table~\ref{resource_analysis} demonstrates a comprehensive comparison of hardware resources and performance metrics for the proposed 4-parallel 1024-point FFT architecture against several state-of-the-art designs. 

The metrics include the number of slice, LUT, flip-flop (FFs), DSP, Block-RAM, maximum clock frequency ($f_{\text{CLK}}$), throughput(Th.), latency (in cycles and microseconds), signal-to-quantization noise ratio (SQNR), power consumption (P), and normalized power (NP).

\begin{table}[h]
\centering
\caption{Comparison of Hardware Resources and Performance for 4-Parallel 1024-Point FFT Implemented on FPGA}
\label{implementation on fPGA}
% 缩小列间距(可选)
\setlength{\tabcolsep}{5pt}
\renewcommand{\arraystretch}{1.5} % 设置行高为默认的1.5倍
% 定义第一列为左对齐、最大宽度 2.2cm(可根据需要调整)
\begin{tabular}{>{\raggedright\arraybackslash}p{1.7cm} c c c c c c}
\toprule
 & \cite{Garrido2018} & \cite{2014ISICGarrido} & \cite{Glittas2016} & \cite{Garrido2021} & \cite{Kaya2024} & Proposed \\
\midrule
Architecture & MDC & MDC & MDC & CM & MSC & MSC \\
Radix & $2^{5}$ & $2^{2}$ & 2 & $2^{5}$ & $2^{5}$ & $2^{5}$\\
WL & 16 & 16 & 16 & 16 & 16 & 16\\
\midrule
Slices & 1420 & 1351 & - & 2631 & 1615 & 1477 \\
LUTs & - & - & 4116 & - & 4682 & 4629 \\
FFs & - & - & 1920 & - & 5910 & 4887 \\
DSPs & 16 & 48 & 72 & 12 & 12 & 12 \\
Block-RAMs & 12 & 12 & 0 & 0 & 4 & 4 \\
\midrule
$f_{\text{CLK}}$ (MHz) & 253 & 227 & 380 & 680 & 420 & 493 \\
Th. (MS/s) & 1012 & 910 & 1520 & 2720 & 1680 & 1820 \\
Latency (cyc.) & 265 & 285 & 767 & 394 & 300 & 307\\
Latency (\si{\micro\second}) & 1.04 & 1.25 & 2.02 & 0.58 & 0.71 &  0.62\\
SQNR (dB) & 40.30 & - & - & - & 50.16 & 49.46 \\
P (W) & - & - & - & 1.68 & 0.98 &  1.16\\
NP ($\frac{\text{mW}}{\text{MHz}}$) & - & - & - & 2.47 & 2.33 &  2.35\\
\bottomrule
\end{tabular}
\end{table} 

\begin{table}[h]
\centering
\caption{Comparison of Resource-Delay Product (RDP) and Power-Delay Product (PDP). PDP = $\text{Power} \cdot \text{T}_{\text{CLK}}$ ($m$W$ \cdot $ns)
, RDP = $\text{Slices} \cdot \text{T}_{\text{CLK}}$ (slices $\cdot$ ns)}
\label{resource_analysis}
% 缩小列间距(可选)
\setlength{\tabcolsep}{3pt} % 减少默认列间距
\renewcommand{\arraystretch}{1.5} % 设置行高为默认的1.5倍
% 根据需要调整每列的宽度以适应半页宽度
\begin{tabular}{
    >{\raggedright\arraybackslash}p{1.6cm}
    *{6}{>{\centering\arraybackslash}m{0.95cm}}
}
\toprule
 & \cite{Garrido2018} & \cite{2014ISICGarrido} & \cite{Glittas2016} & \cite{Garrido2021} & \cite{Kaya2024} & Proposed \\
\midrule
Architecture & MDC & MDC & MDC & CM & MSC & MSC \\
Radix & $2^{5}$ & $2^{2}$ & 2 & $2^{5}$ & $2^{5}$ & $2^{5}$\\
PDP(mW$\cdot$ns) & - & - & - & 2470.61 & 2333.38 & 2352.94 \\
RDP & 5612.69 & 5951.56 & - & 3869.15 & 3845.32 & \textbf{2995.94} \\
\bottomrule
\end{tabular}
\end{table}

\section{Conclusion}
% Summary and future work

% 参考文献(使用 BibTeX)
\bibliographystyle{IEEEtran}
\bibliography{references}  % 假设你的 .bib 文件名为 references.bib

% 如需附录,取消注释以下部分
%\appendices
%\section{Proof of Theorem 1}
%Appendix content here.

% 如需作者简介(带照片)
%\begin{IEEEbiography}[{\includegraphics[width=1in,height=1.25in,clip,keepaspectratio]{photo}}]{Author Name}
%Biography text.
%\end{IEEEbiography}

% 如需作者简介(无照片)
%\begin{IEEEbiographynophoto}{Author Name}
%Biography text.
%\end{IEEEbiographynophoto}

\end{document}